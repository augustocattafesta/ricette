\noindent
\begin{minipage}[t][\textheight][t]{\textwidth}

	\recipetitle{Plumcake}

    \begin{minipage}[t]{0.3\textwidth}
        \vspace{2.5cm}

        \begin{tabularx}{\linewidth}{X}
        {\Large \textsc{Ingredienti}} \\ %\hline \hline
        \end{tabularx}

		\begin{itemize}[label={\ding{32}}, left=2pt]
		\item Impasto
        \begin{itemize}[label={\ding{112}}, left=10pt]
            \item 220 g di farina
            \item 140 g di zucchero
            \item 3 uova
            \item 100 ml di olio di semi
            \item 250 g di yogurt bianco
            \item 16 g di lievito
        \end{itemize}
        \item Facoltativo
        \begin{itemize}[label={\ding{112}}, left=10pt]
        		\item 80/100 g di cioccolato
        \end{itemize}
       	\end{itemize}
    \end{minipage}%
    \hfill
    \begin{minipage}[t]{0.65\textwidth}
        \begin{tabularx}{\linewidth}{X}
        \\
        {\Large \textsc{Procedimento}} \\ \\%\hline \hline \\
        \end{tabularx}
		Montare le uova con lo zucchero. Aggiungere lo yogurt e l'olio e poi le polveri setacciate.\\
		Versare l'impasto nello stampo del plumcake. Cuocere a $170^{\circ}$C per circa 50 minuti. 
        
%     	\includegraphics[valign=t, width=0.5\linewidth]{parts/dolci/images/test}
    \end{minipage}
    
    % Questo spazio spinge la sezione Note verso il basso
    \vspace*{\fill}

    \begin{tabularx}{\linewidth}{X}
    {\Large \textsc{Note}} \\ \hline \hline \\
    \end{tabularx}
	Si può aggiungere del cioccolato tritato grossolanamente nell'impasto, dopo aver incorporato le polveri.
    
\end{minipage}