\noindent
\begin{minipage}[t][\textheight][t]{\textwidth}

	\recipetitle{Cheesecake}

    \begin{minipage}[t]{0.3\textwidth}
        \vspace{2.5cm}

        \begin{tabularx}{\linewidth}{X}
        {\Large \textsc{Ingredienti}} \\ %\hline \hline
        \end{tabularx}

		\begin{itemize}[label={\ding{32}}, left=2pt]
		\item Base
        \begin{itemize}[label={\ding{112}}, left=10pt]
            \item 375 g di biscotti (petit o oswego)
            \item 250 g di burro
        \end{itemize}
        \item Ripieno
        \begin{itemize}[label={\ding{112}}, left=10pt]
			\item 1 kg di Philadelphia
        		\item 400 ml di panna fresca
        		\item 160 g di zucchero
        \end{itemize}
        \item Topping
        \begin{itemize}[label={\ding{112}}, left=10pt]
			\item 350 g di marmellata
        \end{itemize}
        
       	\end{itemize}
    \end{minipage}%
    \hfill
    \begin{minipage}[t]{0.65\textwidth}
        \begin{tabularx}{\linewidth}{X}
        \\
        {\Large \textsc{Procedimento}} \\ \\%\hline \hline \\
        \end{tabularx}
		Sbriciolare o tritare i biscotti, fino a renderli abbastanza fini ma non farinosi. Nel frattempo fondere il burro. Amalgamare i biscotti con il burro e formare la base compattando l'impasto in una teglia a cerniera da 26 cm, con un foglio di carta forno sotto. Far riposare la base in frigo o in freezer.\\
		Montare la philadelphia insieme alla panna e allo zucchero. Versare il composto sulla base dopo averla fatta raffreddare per un po'.\\
		Infine guarnire con la marmellata. Far riposare almeno qualche ora.
        
     	%\includegraphics[valign=t, width=0.5\linewidth]{parts/dolci/images/test}
    \end{minipage}
    
    % Questo spazio spinge la sezione Note verso il basso
    \vspace*{\fill}

    \begin{tabularx}{\linewidth}{X}
    {\Large \textsc{Note}} \\ \hline \hline \\
    \end{tabularx}
	Con 3/4 della dose va bene una teglia a cerniera da 22 cm. Con 350 g di philadelphia si può usare un contenitore di plastica da 1 l
    
\end{minipage}