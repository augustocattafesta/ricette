\documentclass[../../main.tex]{subfiles}
%\graphicspath{{\subfix{images/}}}


\ExplSyntaxOn

\NewDocumentCommand{\completa}{m m m m m}{
\noindent
\begin{minipage}[t][\textheight][t]{\textwidth}

	\ricetta{comp}
    
    \begin{minipage}[t]{0.3\textwidth}
        \includegraphics[valign=t, width=\linewidth]{images/test.jpg}
        \vspace{0.5em}

        \begin{tabularx}{\linewidth}{X}
        {\Large \textsc{Ingredienti}} \\ \hline \hline
        \end{tabularx}

        \begin{itemize}[label={\ding{112}}, left=0pt]
            \item 300 g di farina 00
            \item 100 g di zucchero
            \item 2 uova intere
            \item 70 ml di olio di semi
        \end{itemize}
    \end{minipage}%
    \hfill
    \begin{minipage}[t]{0.65\textwidth}
        \begin{tabularx}{\linewidth}{X}
        \\
        {\Large \textsc{Procedimento}} \\ \hline \hline \\
        \end{tabularx}

        Mischiare farina e zucchero in una ciotola. Fare la fontana e mettere l'olio e le uova precedentemente sbattute al centro. Mischiare con la forchetta unendo poco a poco tutta la farina. Lavorare un po' con le mani su un tagliere e, se serve, aggiungere un po' di farina in modo che l'impasto non attacchi più.
    \end{minipage}

    % Questo spazio spinge la sezione Note verso il basso
    \vspace*{\fill}

    \begin{tabularx}{\linewidth}{X}
    {\Large \textsc{Note}} \\ \hline \hline \\
    \end{tabularx}
    Si ottengono circa 600 g di impasto, abbondanti per una crostata in una teglia da 26 cm (bastano anche 450 g per la base). Con l'impasto che avanza escono circa 45 biscotti a forma di stellina piccola (1 teglia).
\end{minipage}

}

\ExplSyntaxOff
