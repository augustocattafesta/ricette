\noindent
\begin{minipage}[t][\textheight][t]{\textwidth}

	\recipetitle{Torta 4 Dicembre}

    \begin{minipage}[t]{0.3\textwidth}
        \vspace{2.5cm}

        \begin{tabularx}{\linewidth}{X}
        {\Large \textsc{Ingredienti}} \\ %\hline \hline
        \end{tabularx}

        \begin{itemize}[label={\ding{112}}, left=10pt]
			\item 250 g di farina
			\item 180 g di zucchero
			\item 3 uova
			\item 250 g di ricotta
			\item 80 g di burro
			\item 16 g di lievito
			\item 80 g di cioccolato
        \end{itemize}

    \end{minipage}%
    \hfill
    \begin{minipage}[t]{0.65\textwidth}
        \begin{tabularx}{\linewidth}{X}
        \\
        {\Large \textsc{Procedimento}} \\ \\%\hline \hline \\
        \end{tabularx}
		Montare a neve gli albumi. Montare i tuorli con lo zucchero. Nel frattempo fondere il burro. Versare il burro fuso e poi la ricotta e continuare a mischiare.\\
		Incorporare delicatamente gli albumi e in seguito le farina e lievito setacciati. Unire il latte e infine versare il cioccolato tritato grossolanamente.\\
		Versare l'impasto in uno stampo da 22 cm. Cuocere in forno statico a $180^{\circ}$ per circa 30 minuti.
        
%     	\includegraphics[valign=t, width=0.5\linewidth]{parts/dolci/images/test}
    \end{minipage}
    
    % Questo spazio spinge la sezione Note verso il basso
    \vspace*{\fill}

    \begin{tabularx}{\linewidth}{X}
    {\Large \textsc{Note}} \\ \hline \hline \\
    \end{tabularx}
	Servire con una spolverata di zucchero a velo.
    
\end{minipage}