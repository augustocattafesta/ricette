\documentclass[../../main.tex]{subfiles}
%\graphicspath{{\subfix{images/}}}
\begin{document}

\ricetta{Pasta Frolla}

\noindent
\begin{minipage}[t]{0.3\textwidth}
    \includegraphics[valign=T, width=1\linewidth]{images/test.jpg}  % Più grande
    \section*{Ingredienti}
    \begin{itemize}
        \item 300 g di farina 00
        \item 100 g di zucchero
        \item 2 uova intere
        \item 70 ml di olio di semi
    \end{itemize}
\end{minipage}%
\hfill
\begin{minipage}[t]{0.65\textwidth}
\section*{Procedimento}
Mischiare farina e zucchero in una ciotola. Fare la fontana e mettere l'olio e le uova precedentemente sbattute al centro. Mischiare con la forchetta unendo poco a poco tutta la farina. Lavorare un po' con le mani su un tagliere e se serve aggiungere un po' di farina, in modo che l'impasto non attacchi più.
\end{minipage}
\vspace*{\fill}
\section*{Note}	% mettere in fondo alla pagina
Si ottengono circa 600 g di impasto, abbondanti per una crostata in una teglia da 26  cm (bastano anche 450 g per la base). Con l'impasto che avanza escono circa 45 biscotti a forma di stellina piccola (1 teglia).
\vspace*{\fill}

\end{document}