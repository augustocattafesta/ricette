\noindent
\begin{minipage}[t][\textheight][t]{\textwidth}

	\recipetitle{Pasta Frolla}

    \begin{minipage}[t]{0.3\textwidth}
        \vspace{2.5cm}

        \begin{tabularx}{\linewidth}{X}
        {\Large \textsc{Ingredienti}} \\ %\hline \hline
        \end{tabularx}
        \begin{itemize}[label={\ding{112}}, left=10pt]
            \item 300 g di farina 00
            \item 100 g di zucchero
            \item 2 uova intere
            \item 70 ml di olio di semi
        \end{itemize}

    \end{minipage}%
    \hfill
    \begin{minipage}[t]{0.65\textwidth}
        \begin{tabularx}{\linewidth}{X}
        \\
        {\Large \textsc{Procedimento}} \\ \\%\hline \hline \\
        \end{tabularx}
		Mischiare farina e zucchero in una ciotola. Fare la fontana, versare olio e uova sbattute al centro e mischiare con la forchetta, unendo tutto tutta la farina poco a poco.\\
		Lavorare un po' con le mani, aggiungendo, se necessario, un po' di farina fino a che l'impasto non è liscio e non attacca più.\\
		Cuocere a $180^{\circ}$ per circa 40 minuti.
        
     	\includegraphics[valign=t, width=0.5\linewidth]{images/dolci/test}
    \end{minipage}
    
    % Questo spazio spinge la sezione Note verso il basso
    \vspace*{\fill}

    \begin{tabularx}{\linewidth}{X}
    {\Large \textsc{Note}} \\ \hline \hline \\
    \end{tabularx}
	Si ottengono circa 600 g di impasto, abbondanti per una crostata in una teglia da circa 26 cm ($\approx$ 450 g per la base). Con l'impasto avanzante dalla crostata si possono fare circa 45 biscotti a forma di stellina (la più piccola).
    
\end{minipage}