\documentclass[12pt, oneside]{book}

\usepackage{multicol}
\usepackage[a4paper, margin=2cm]{geometry}
\usepackage{graphicx}
\usepackage[export]{adjustbox}
\usepackage{subfiles}
\usepackage{tabularx} % nel preambolo
\newcolumntype{Y}{>{\centering\arraybackslash}X}

\usepackage{adjustbox}

% Pacchetti di prova
\usepackage{lipsum}
\usepackage{titlesec}

% Definiamo una macro per memorizzare il titolo della parte
\newcommand{\parttitle}{}

% Ridefiniamo \part per salvare il titolo in \parttitle
\let\oldpart\part
\renewcommand{\part}[1]{%
  \def\parttitle{#1}%
  \oldpart{#1}%
}


\newcommand{\ricetta}[1]{\addcontentsline{toc}{section}{#1}\begin{tabularx}{1\linewidth}{Y Y Y}
 & {\Large \textsc{\parttitle}} &  \\ \hline \hline \\
 & \makebox[\linewidth][c]{{\LARGE \textsc{#1}}} \\ &  \\ \hline \hline
\end{tabularx}}


\begin{document}

\title{Raccolta di Ricette}
\author{A. Cattafesta, C. Selicato}
\date{}

\maketitle
\tableofcontents
\part{Dolci}

%\addcontentsline{toc}{section}{Pasta frolla}
\subfile{parts/dolci/frolla}

\addcontentsline{toc}{section}{Crostata}
\subfile{parts/dolci/crostata_marmellata}

\newpage

\ricetta{Crostata alla Marmellata}




\end{document}